%====================
% Intent Letter
%====================

%\hfill \break

\noindent\hbox to 0.5\textwidth{}


I have had well over a hundred teachers and mentors in my life, from kindergarten to graduate school.  The majority I do not recall at all, a few I remember with a lack of fondness and an elite few have shaped me immensely and I quietly thank them in my head from time to time.  Later in life as I began teaching while pursuing my masters and doctoral degrees, and as a colleague and a leader in my corporate life,  I pondered the hallmarks of an effective educator and began to formulate my own set of philosophies for what an effective one should look like, taking inspiration and incorporating the wisdom of the mentors who positively influenced me. \\
\vspace{2ex}
I believe that central to being an effective educator means building good relationships with students and setting a tone that fosters mutual respect and trust.  Mr. Ludlow was my high school band director and until this day, remains my favorite teacher of all.  He had the gift of making everyone feel heard and valued, engaging with each and every student sincerely.  In my classrooms I constantly strive to create an environment that allows the students to freely share their ideas and thoughts without judgment and encourage them to discuss their experiences as it relates to the material that we are studying.  Of course this means being open to diversity and a wide range of opinions and respecting one another; I want my own classroom to be less of a class and more of a community, where we are all sharing ideas and learning together. \\
\vspace{2ex}
Additionally, I believe outstanding educators conduct themselves with professionalism, fairness and consistency but share an infectious enthusiasm and passion for learning.  I hold my students to a high standard and offer ample constructive feedback, designed to coax critical thinking skills and bolster strong work ethics and integrity, but always in a respectful manner.  Equally important is the instructor's passion for the subject.  One of my most enjoyable classes I took in college was studying Shakespeare's works.  Going into it, I was not looking forward to it, for it was merely to fulfill a requirement, however, by the end of the semester, I had thoroughly enjoyed the class and had learned so much.  The key was an enthusiastic professor; it was evident that she had so much passion for the subject.  I go into each class with high energy and aim to spread love for the subjects that I teach. \\
\vspace{2ex}
Beyond imparting academic knowledge, I approach my role with a bigger picture in mind.  I strive to instill a sense of self-awareness and kindness, impart practical skills necessary for real-life applications especially in this ever-changing fast-paced global economy, and help instill a lifelong love and pursuit of learning and encourage them to stay curious.  I myself embrace continuous learning by staying abreast of current discoveries in my fields of study.  Perhaps Socrates captured my sentiment best: ``Education is the kindling of a flame, not the filling of a vessel.''
