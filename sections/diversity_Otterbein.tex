%====================
% Diversity Statement
%====================

%\hfill \break

\noindent\hbox to 0.5\textwidth{}

I grew up in an extremely rural America without a single minority group in sight.  It was not until I entered college at Ohio State University, that I began to see and interact with people who do not share the same background as I.  After a while, I began to realize that I had taken for granted some of the inherent privileges I had based solely on the color of my skin, gender and sexual orientation.  At that point, I made a conscious decision to embrace and celebrate everyone's differences and educate myself on the various cultures of people who look and think differently than I do.  One of the ways I decided to take action was to volunteer to teach English to a group of Somali refugees for a non-profit organization; I learned a great deal about them and their culture as I gained their trust.  Through taking genuine interest in people and through life experiences, I began to notice the existence of disparity and therefore disadvantages faced by those who do not share my background; this motivated me to want to become an agent of change to support equity in all facets of life.  \\
\vspace{2ex}
When I entered the workforce upon finishing my doctoral degree I became even more mindful and committed to advocating for fairness across all humans.  So when an opportunity presented itself for me to spearhead an effort to raise awareness and acceptance of our LGBTQ+ community, I jumped on it. I had the privilege of leading my team of fifty individuals through an initiative centered on supporting a coworker who underwent gender-affirming surgery.  I led weekly workshops and training sessions where team members were encouraged to express their thoughts and concerns, and information to dispel misconceptions were discussed.  By leading with compassion, where questions were welcomed and curiosity was met with open dialogue, we were able to raise awareness and sensitivity and collectively moved toward a supportive workplace that embraces diversity. \\
\vspace{2ex}
I believe workforce equity and diversity is a topic that needs vigilant and continual evaluation and conversation.  After more than twenty years in the business analytics and data science industry, it was evident that there was a disproportionately fewer number of women and members of underrepresented ethnic minorities in this workspace.  I believe this lack of diversity is in part due to gender and cultural stereotyping in education.  My company's hiring pool focused on candidates that majored in data science, economics, finance, chemical engineering, computer science, electrical engineering, industrial engineering, mathematics, and statistics; these are all majors that are typically not dominated by women and minorities.  \\
\vspace{2ex}
Change does not happen overnight, but it is critical that we as a society address the inequalities experienced by women and underrepresented minorities in certain areas of the workforce.  Universities and corporations can do better by implementing equal opportunity practices and laying the groundwork and creating a climate to build confidence and ensure success in these underrepresented groups.  If I am hired as a faculty member at Otterbein University, I will be committed in my efforts to mentor these underrepresented students, and establish a positive culture, and to further promote diversity. \\
\vspace{2ex}
I wholeheartedly believe in the far-reaching impact and the benefits of diversity, equity and inclusivity; I would be honored to be a part of an institution such as Otterbein University that embraces these values and plays such a crucial role in shaping our world.
